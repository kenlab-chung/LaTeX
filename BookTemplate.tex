\documentclass[utf8]{book}
\usepackage{titletoc}
\usepackage{titlesec}
\usepackage{ctexcap}
\usepackage[b5paper,text={125mm,195mm},centering,left=1in,right=1in,top=1in,bottom=1in]{geometry}
\usepackage[]{geometry}
\usepackage{imakeidx}
\usepackage{multicol}
%% 首先,如果要添加目录,则需要在(希望)目录出现的地方使用命令\tableofcontents;如果想要为目录添加超链接实现跳转功能,则需要使用\usepackage{hyperref}命令,但是这时目录上会有红色方框,如果要去除红色方框,则需要给hyperref命令添加相应的参数,因此需要使用\usepackage[hidelinks]{hyperref}命令。
%\usepackage{hyperref}
\usepackage[hidelinks]{hyperref}

%插入c代码
\usepackage{listings}
\usepackage{xcolor}
\lstset{
	numbers=left, 
	numberstyle= \tiny, 
	keywordstyle= \color{ blue!70},
	commentstyle= \color{red!50!green!50!blue!50}, 
	frame=shadowbox, % 阴影效果
	rulesepcolor= \color{ red!20!green!20!blue!20} ,
	escapeinside=``, % 英文分号中可写入中文
	xleftmargin=2em,xrightmargin=0em, aboveskip=1em,
	framexleftmargin=2em
} 

% 插入python代码
\usepackage{graphicx}
\usepackage{pythonhighlight}

\makeindex
\bibliographystyle{plain}
\begin{document}
	\title{\heiti 中国植物多样性地理图集}
	\author{\fangsong 中国科学院植物研究所 编著}
	\date{2018年1月}
	
	\frontmatter
	\maketitle
	
	\chapter{序 I}
	
	这部分是序言这部分是序言这部分是序言这部分是序言这部分是序言这部分是序言这部分是序言这部分是序言这部分是序言这部分是序言这部分是序言这部分是序言这部分是序言这部分是序言这部分是序言这部分是序言这部分是序言这部分是序言这部分是序言这部分是序言这部分是序言这部分是序言这部分是序言这部分是序言这部分是序言
	
	\chapter{序 II}
	
	这部分是序言这部分是序言这部分是序言这部分是序言这部分是序言这部分是序言这部分是序言这部分是序言这部分是序言这部分是序言这部分是序言这部分是序言这部分是序言这部分是序言这部分是序言这部分是序言这部分是序言这部分是序言这部分是序言这部分是序言这部分是序言这部分是序言这部分是序言这部分是序言这部分是序言这部分是序言这部分是序言这部分是序言这部分是序言这部分是序言这部分是序言这部分是序言这部分是序言这部分是序言这部分是序言这部分是序言这部分是序言这部分
	
	\chapter{前~言}
	
	这部分是前言这部分是前言这部分是前言这部分是前言这部分是前言这部分是前言这部分是前言这部分是前言这部分是前言这部分是前言这部分是前言这部分是前言这部分是前言这部分是前言这部分是前言这部分是前言这部分是前言这部分是前言这部分是前言这部分是前言这部分是前言这部分是前言这部分是前言这部分是前言这部分是前言这部分是前言这部分是前言这部分是前言这部分是前言这部分是前言这部分是前言这部分是前言这部分是前言这部分是前言这部分是前言这部分是前言这部分是前言这部分
	
	这部分是前言
	
	\renewcommand\contentsname{目~录}
	\tableofcontents
	
	\mainmatter
	
	\part{总论}
	
	\chapter{中国植物调查的历史}
	
	\section{一级节标题}
	
	\subsection{二级节标题}
	
	\subsubsection{三级节标题}
	
	\begin{multicols}{2}
		
		清朝中期, 外国人就开始中国采集植物标本。清朝中期, 外国人就开始中国采集植物标本。清朝中期, 外国人就开始中国采集植物标本。清朝中期, 外国人就开始中国采集植物标本。清朝中期, 外国人就开始中国采集植物标本。清朝中期, 外国人就开始中国采集植物标本。
	
	\end{multicols}
	\lstset{language=C}
	\begin{lstlisting}
	#include <stdio.h>
	#include <stdbool.h>
	#include <ctype.h>
	
	#define SIZE 26
	
	int main (int argc, char *argv[])
	{
	int array[SIZE];
	int i;
	char c;
	for (i = 0; i < SIZE; i++)
	array[i] = 0;
	while ((c = getchar ()) != EOF)
	{
	if (isupper (c))
	{
	array[c - 'A']++;
	}
	}
	for (i = 0; i < 26; i++)
	printf ("%c:%5d\n", (char) ('A' + i), array[i]);
	return 0;
	}
	\end{lstlisting}	
	\chapter{植物区系分区}
	
	\section{一级节标题}
	
	\subsection{二级节标题}
	
	\subsubsection{三级节标题}
	
	\begin{multicols}{2}
		和标本采集植物调查和标本采集植物调查和标本采集植物调查和标本采集植物调查和标本采集植物调查和标本采集植物调查和标本采集植物调查和标本采集植物调查和标本采集植物调查和标本采集植物调查和标本采集植物调查和标本采集植物调查和标本采集植物调查和标本采集植物调查和标本采集植物调查和标本采集植物调查和标本采集植物调查和标本采集植物调查和标本采集植物调查和标本采集植物\index{调查}和标本采集植物调查和标本采集植物调查和标本采集植物调查和标本采集
		
	\end{multicols}
	
	\part{植物多样性分区概述}
	
	\chapter{东北地区}
	
	\begin{multicols}{2}
		Lorem ipsum dolor sit amet, consectetur adipisicing elit, sed do eiusmod tempor incididunt ut labore et dolore magna aliqua. Ut enim ad minim veniam, quis nostrud exercitation ullamco laboris nisi ut aliquip ex ea commodo consequat. Duis aute irure dolor in \index{reprehenderit} in voluptate velit esse cillum dolore eu fugiat nulla pariatur. Excepteur sint occaecat cupidatat non proident, sunt in culpa qui officia deserunt mollit anim id est laborum.
	\end{multicols}
	
	\chapter{华北地区}
	\begin{multicols}{2}
		Lorem ipsum dolor sit amet, consectetur adipisicing elit, sed do eiusmod tempor incididunt ut labore et dolore magna aliqua. Ut enim ad minim veniam, quis nostrud exercitation ullamco laboris nisi ut aliquip ex ea commodo consequat. Duis aute irure dolor in reprehenderit in voluptate velit esse cillum dolore eu fugiat nulla pariatur. Excepteur sint occaecat cupidatat non proident, sunt in culpa qui officia deserunt mollit anim id est laborum.
	\end{multicols}
	
	\include{angiosperms}
	
	\appendix
	
	\chapter{APG分类系统的科名}
	
	内容附录内容附录内容附录内容附录内容附录内容附录内容附录内容附录内容附录内容附录内容附录内容附录内容附录内容附录内容附录内容附录内容附录内容文内容正文内容正文内容正文\index{内容}
	
	内容附录内容附录内容附录内容附录内容附录内容附录内容附录内容附录内容附录内容附录内容附录内容附录内容附录内容附录内容附录内容附录内容附录内容附录内容附录内容附录内容正文内容正文\cite{DK1}.
		\section{插入C代码}
	\lstset{language=C}
	\begin{lstlisting}
#include <stdio.h>
#include <stdbool.h>
#include <ctype.h>

#define SIZE 26

int main (int argc, char *argv[])
{
  int array[SIZE];
  int i;
  char c;
  for (i = 0; i < SIZE; i++)
    array[i] = 0;
  while ((c = getchar ()) != EOF)
  {
    if (isupper (c))
    {
      array[c - 'A']++;
    }
  }
  for (i = 0; i < 26; i++)
    printf ("%c:%5d\n", (char) ('A' + i), array[i]);
  return 0;
}
	\end{lstlisting}	
	\section{插入C++代码}
	\lstset{language=C++}   
	\begin{lstlisting}
#include<iostream>
using namespace std;
int main(){
  cout<<"hello world!"<<endl;
}
	\end{lstlisting}
	
	\section{插入java代码}	
	%	\lstset{language=Java}
	\begin{lstlisting} [language=Java]
package information;

import java.io.File;
import java.io.FileOutputStream;
import java.io.IOException;
import java.io.OutputStream;
import java.util.ArrayList;

public final class Graph {
  private final static String curDir =
	System.getProperty("user.dir")
	+System.getProperty("file.separator")
	+"output";
 private final static String sysLineSep =
	System.getProperty("line.separator");
 private final static byte[] lsByteArr =
 	sysLineSep.getBytes();
 private ArrayList<String> edges = new ArrayList<>();
 private ArrayList<String> nodes = new ArrayList<>();

 public Graph() {
  super();
 }

 public void addEdge(int from,int to,Character c)
 {
  this.addEdge(String.valueOf(from),
  	String.valueOf(to), 
  	String.valueOf(c));
 }
 public void addEdge(String from,String to, String c)
 {
  edges.add(from+" -> to"
 	+String.format("[label=\"%s\"]", c));
 }
}

\end{lstlisting}
	
	\section{插入python代码}	
\begin{python}
#
from pyx import *
	
g = graph.graphxy(width=8)
g.plot(graph.data.function("y(x)=sin(x)/x", min=-15, max=15))
g.writePDFfile("function")
print r'\includegraphics{function}'
\end{python}
	\chapter{APG分类系统的科名1}	
	内容附录内容附录内容附录内容附录内容附录内容附录内容附录内容附录内容附录内容附录内容附录内容附录内容附录内容附录内容附录内容附录内容附录内容文内容正文内容正文内容正文\index{内容}
	
	内容附录内容附录内容附录内容附录内容附录内容附录内容附录内容附录内容附录内容附录内容附录内容附录内容附录内容附录内容附录内容附录内容附录内容附录内容附录内容附录内容正文内容正文\cite{DK1}.
	\chapter{APG分类系统的科名1}	
	内容附录内容附录内容附录内容附录内容附录内容附录内容附录内容附录内容附录内容附录内容附录内容附录内容附录内容附录内容附录内容附录内容附录内容文内容正文内容正文内容正文\index{内容}
	
	内容附录内容附录内容附录内容附录内容附录内容附录内容附录内容附录内容附录内容附录内容附录内容附录内容附录内容附录内容附录内容附录内容附录内容附录内容附录内容附录内容正文内容正文\cite{DK1}.
	\renewcommand\indexname{索~~引}
	\printindex
	\addcontentsline{toc}{chapter}{索~引}
	
	\backmatter
	
	\addcontentsline{toc}{chapter}{参考文献}
	
	\begin{thebibliography}{参考文献}
		\bibitem[Knuth1 et al. 1997]{DK1} D. Knuth, T.A.O.C.P. , Vol. 1, Addison-Wesley, 1997.
		\bibitem[Knuth2]{DK2} D. Knuth, T.A.O.C.P. , Vol. 2, Addison-Wesley, 1997.
		\bibitem[TONG YH 2014]{TONG} TONG YH, PANG KS, XIAN NH, 2014. Carpinus insularis (Betulaceae), A new species from Hong Kong [J]. J Trop Subtrop Bot, 22(2): 121-124. [童毅华, 彭权森, 夏念和,2014. 香港桦木科一新种——香港鹅耳枥 [J]. 热带亚热带植物学报,22(2):121-124.]
		\bibitem[XIA NH 2008]{XIA} XIA NH, DENG YF, YIP KL, 2008. Syzygium impressum (Myrtaceae), A new species from Hong Kong [J]. J Trop Subtrop Bot, 16(1):19-22. [夏念和,邓云飞,叶国梁,2008. 香港桃金娘科一新种-凹脉赤楠 [J].热带亚热带植物学报,16(1):19-22.]
	\end{thebibliography}
	
	\chapter{后~~记}
	
	后记内容后记内容后记内容后记内容后记内容后记内容后记内容后记内容后记内容后记内容后记内容后记内容后记内容后记内容后记内容后记内容后记内容后记内容后记内容后记内容后记内容后记内容后记内容后记内容后记内容后记内容后记内容后记内容后记内容后记内容后记内容后记内容后记内容后记内容后记内容后记内容后记内容后记内容后记内容后记内容后记内容后记内容后记内容后记内容后记内容后记内容后记内容后记内容后记内容后记内容后记内容后记内容后记内容后记内容后记内容后记内容后记内容后记内容后记内容后记内容后记内容后记内容后记内容后记内容后记内容
	
	\begin{flushright}
		作~~者~~~~~~~~~
		
		2018年1月~~~~~
	\end{flushright}
	
\end{document}
